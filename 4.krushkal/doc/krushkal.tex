%%%%%%%%%%%%%%%%%%%%%%%%%%%%%%%%%%%%%%%%%
% Journal Article
% Data Structure and Algorithm
% Practical 4: Implementing Krushkal's Algorithm for minimum spanning tree with disjoint set data structure
%
% Gahan M. Saraiya
% 18MCEC10
%
%%%%%%%%%%%%%%%%%%%%%%%%%%%%%%%%%%%%%%%%%
%----------------------------------------------------------------------------------------
%       PACKAGES AND OTHER DOCUMENT CONFIGURATIONS
%----------------------------------------------------------------------------------------
\documentclass[paper=letter, fontsize=12pt]{article}
\usepackage[english]{babel} % English language/hyphenation
\usepackage{amsmath,amsfonts,amsthm} % Math packages
\usepackage[utf8]{inputenc}
\usepackage{float}
\usepackage{lipsum} % Package to generate dummy text throughout this template
\usepackage{blindtext}
\usepackage{graphicx} 
\usepackage{caption}
\usepackage{subcaption}
\usepackage[sc]{mathpazo} % Use the Palatino font
\usepackage[T1]{fontenc} % Use 8-bit encoding that has 256 glyphs
\usepackage{bbding}  % to use custom itemize font
\linespread{1.05} % Line spacing - Palatino needs more space between lines
\usepackage{microtype} % Slightly tweak font spacing for aesthetics
\usepackage[hmarginratio=1:1,top=32mm,columnsep=20pt]{geometry} % Document margins
\usepackage{multicol} % Used for the two-column layout of the document
%\usepackage[hang, small,labelfont=bf,up,textfont=it,up]{caption} % Custom captions under/above floats in tables or figures
\usepackage{booktabs} % Horizontal rules in tables
\usepackage{float} % Required for tables and figures in the multi-column environment - they need to be placed in specific locations with the [H] (e.g. \begin{table}[H])
\usepackage{hyperref} % For hyperlinks in the PDF
\usepackage{lettrine} % The lettrine is the first enlarged letter at the beginning of the text
\usepackage{paralist} % Used for the compactitem environment which makes bullet points with less space between them
\usepackage{abstract} % Allows abstract customization
\renewcommand{\abstractnamefont}{\normalfont\bfseries} % Set the "Abstract" text to bold
\renewcommand{\abstracttextfont}{\normalfont\small\itshape} % Set the abstract itself to small italic text
\usepackage{titlesec} % Allows customization of titles

%----------------------------------------------------------------------------------------
%       packages to draw graph
%---------------------------------------------------------
\usepackage{tikz}

\usepackage{verbatim}
\usetikzlibrary{arrows,shapes}
%----------------------------------------------------------------------------------------


\renewcommand\thesection{\Roman{section}} % Roman numerals for the sections
\renewcommand\thesubsection{\Roman{subsection}} % Roman numerals for subsections

\titleformat{\section}[block]{\large\scshape\centering}{\thesection.}{1em}{} % Change the look of the section titles
\titleformat{\subsection}[block]{\large}{\thesubsection.}{1em}{} % Change the look of the section titles
\newcommand{\horrule}[1]{\rule{\linewidth}{#1}} % Create horizontal rule command with 1 argument of height
\usepackage{fancyhdr} % Headers and footers
\pagestyle{fancy} % All pages have headers and footers
\fancyhead{} % Blank out the default header
\fancyfoot{} % Blank out the default footer

\fancyhead[C]{Institute of Technology, Nirma University $\bullet$ October 2018} % Custom header text

\fancyfoot[RO,LE]{\thepage} % Custom footer text
%----------------------------------------------------------------------------------------
%       TITLE SECTION
%----------------------------------------------------------------------------------------
\title{\vspace{-15mm}\fontsize{24pt}{10pt}\selectfont\textbf{Practical 4 \\ Implementing Krushkal's Algorithm for minimum spanning tree with disjoint set data structure}} % Article title
\author{
\large
{\textsc{Gahan Saraiya, 18MCEC10 }}\\[2mm]
%\thanks{A thank you or further information}\\ % Your name
\normalsize \href{mailto:18mcec10@nirmauni.ac.in}{18mcec10@nirmauni.ac.in}\\[2mm] % Your email address
}
\date{}
\hypersetup{
	colorlinks=true,
	linkcolor=blue,
	filecolor=magenta,      
	urlcolor=cyan,
	pdfauthor={Gahan Saraiya},
	pdfcreator={Gahan Saraiya},
	pdfproducer={Gahan Saraiya},
}
\usepackage{minted} % for highlighting code sytax
\usepackage{xcolor}
\definecolor{LightGray}{gray}{0.9}

\setminted[text]{
	frame=lines, 
	breaklines,
	baselinestretch=1.2,
	bgcolor=LightGray,
	%	fontsize=\small
}
%----------------------------------------------------------------------------------------
\usepackage[utf8]{inputenc}
\usepackage[english]{babel}

\begin{document}
\maketitle % Insert title
\thispagestyle{fancy} % All pages have headers and footers

\newcommand*\tick{\item[\Checkmark]}
\newcommand*\arrow{\item[$\Rightarrow$]}
\newcommand*\fail{\item[\XSolidBrush]}

\section{Introduction}
\paragraph{}
Aim of this practical is to implement Krushkal's algorithm in \textit{C Language} and using disjoint set data structure to detect cycle.

\section{Kruskal's Algorithm}
\begin{itemize}
	\item Greedy Algorithm
	\item Finds out Minimum Spanning Tree (MST) for a connected weighted graph
	\item finds subset of vertex having total weight of all edges in tree is minimized.
	\item Directly based on MST property.
\end{itemize}

\section{Program Logic}
\begin{enumerate}
	\item Consider input graph in adjacency matrix 
	\item Create list of edges for given graph, with their weights.
	\item Initialize array representing Disjoint set Data Structure
	\item Sort edge list in ascending order.
	\item \label{start} Pick Edge with minimum weight (Top item from edge list)
	\item Remove Edge from edge list
	\item \label{end} Check for cycle formation if edge selected, if forms cycle then discard it otherwise add it to the list of edges for minimum spanning tree.
	\item Repeat \ref{start} to \ref{end} until list of edges over or MST (minimum spanning tree) contains $ total\_edges - 1$ edges.
	
\end{enumerate}

% Declare layers
\pgfdeclarelayer{background}
\pgfsetlayers{background,main}
\tikzstyle{vertex}=[circle,fill=black!25,minimum size=20pt,inner sep=0pt]
\tikzstyle{selected vertex} = [vertex, fill=red!24]
\tikzstyle{edge} = [draw,thick,-]
\tikzstyle{weight} = [font=\small]
\tikzstyle{selected edge} = [draw,line width=5pt,-,red!50]
\tikzstyle{ignored edge} = [draw,line width=5pt,-,black!20]

%\begin{figure}
%	\begin{tikzpicture}[scale=1.8, auto,swap]
%	% Draw a 7,11 network
%	% First we draw the vertices
%	\foreach \pos/\name in {{(0,2)/a}, {(2,1)/b}, {(4,1)/c},
%		{(0,0)/d}, {(3,0)/e}, {(2,-1)/f}, {(4,-1)/g}}
%	\node[vertex] (\name) at \pos {$\name$};
%	% Connect vertices with edges and draw weights
%	\foreach \source/ \dest /\weight in {b/a/7, c/b/8,d/a/5,d/b/9,
%		e/b/7, e/c/5,e/d/15,
%		f/d/6,f/e/8,
%		g/e/9,g/f/11}
%	\path[edge] (\source) -- node[weight] {$\weight$} (\dest);
%	% Start animating the vertex and edge selection. 
%	\foreach \vertex / \fr in {d/1,a/2,f/3,b/4,e/5,c/6,g/7}
%	\path<\fr-> node[selected vertex] at (\vertex) {$\vertex$};
%	% For convenience we use a background layer to highlight edges
%	% This way we don't have to worry about the highlighting covering
%	% weight labels. 
%	\begin{pgfonlayer}{background}
%	\pause
%	\foreach \source / \dest in {d/a,d/f,a/b,b/e,e/c,e/g}
%	\path<+->[selected edge] (\source.center) -- (\dest.center);
%	\foreach \source / \dest / \fr in {d/b/4,d/e/5,e/f/5,b/c/6,f/g/7}
%	\path<\fr->[ignored edge] (\source.center) -- (\dest.center);
%	\end{pgfonlayer}
%	\end{tikzpicture}
%\end{figure}

\section{Implementation}

\inputminted[frame=lines, breaklines, linenos]{c}{../krushkal_disjoint_ds.c}

\section*{Output}
\inputminted[frame=lines, breaklines]{text}{../krushkal_output.txt}



%\section{Summary}
%\begin{itemize}
%	\tick If directory fits in memory then point query requires only $ 1 $ disk access
%\end{itemize}
%
%\setlength{\tabcolsep}{10pt} % Default value: 6pt
%\renewcommand{\arraystretch}{1.5} % Default value: 1
%\begin{table}[!ht]
%\begin{flushleft}
%\centering
%\caption*{Time Complexity of Extendible Hashing for single record access}
%\begin{tabular}{ l  c c }
%
%\hline
%{} &\multicolumn{2}{ c }{\textbf{Condition}}\\ 
%\cline{1-3}
%\textbf{} & \textbf{directory size < memory size} & \textbf{directory size > memory size} \\
%\hline
%Access & 1 & 2 \\ 
% \hline
%\end{tabular}
%\end{flushleft}
%\end{table}
%
%
%\subsection*{Space Complexity}
%\begin{table}[H]
%	\begin{tabular}{r l}
%		R & Number of records \\
%		B & Block Size \\
%		N & Number of blocks \\
%	\end{tabular}
%\end{table}
%\paragraph{Space Utilization}
%\begin{equation}
%	\frac{R}{B \times N}
%\end{equation}
%\paragraph{Average Utilization} $\ln{2} \approx 0.69$
%----------------------------------------------------------------------------------------
%\end{multicols}
\end{document}
